\documentclass[msc,    % ou [msc] para dissertações 
  % hidetitle,         % de mestrado. Para propostas ou
  hidecover,           % projetos, usar [phd,project],
  hidefc,              % [msc,proposal], etc.
  hideapproval,
  hideack,
  a4paper
  ]{ppgccufmg} 

\usepackage[brazil]{babel}        % se o documento for em português, OU
%\usepackage[english]{babel}      % se o documento for em inglês
\usepackage[utf8x]{inputenc}
\PrerenderUnicode{ã}
\usepackage[T1]{fontenc}
\usepackage{type1ec}
\usepackage{graphicx}
\usepackage{multirow}
\usepackage{listings} % escrever trechos de códigos
\usepackage{xcolor}
\lstset { %
  language=C++,
  backgroundcolor=\color{black!5}, % set backgroundcolor
  basicstyle=\footnotesize,% basic font setting
  showstringspaces=false,
}
\usepackage{subcaption} % subfigures
\usepackage{amssymb,amsmath} % equações

\usepackage[portuguese,
  bookmarks=true,
  bookmarksnumbered=true,
  linktocpage,
  colorlinks,
  citecolor=black,
  urlcolor=black,
  linkcolor=black,
  filecolor=black
  ]{hyperref}
\usepackage[square]{natbib}
\usepackage{ltxtable, tabularx, longtable}
\usepackage{booktabs}% http://ctan.org/pkg/booktabs
\newcommand{\tabitem}{\textbullet~}

\begin{document}

% O comando a seguir, \ppgccufmg, provê todas as informações relevantes para a
% classe ppgccufmg. Por favor, consulte a documentação para a descrição de
% cada chave.

% Um exemplo para documentos em português é apresentado a seguir:
\ppgccufmg{
  title={Método Automático de Contagem Volumétrica de Veículos baseado em Visão Computacional},
  authorrev={Ferreira Bailão, Arthur},
  % cutter={D1234p},
  % cdu={519.6*82.10},
  university={Universidade Federal de Minas Gerais},
  course={Engenharia de Controle e Automação},
  address={Belo Horizonte},
  date={2014-11},
  keywords={Visão Computacional, Engenharia de Transporte, Contagem Volumétrica},
  advisor={Prof. Hermes Aguiar Magalhães},
  supervisor={Profª. Leise Kelli de Oliveira},
%  approval={img/approvalsheet.eps},
%  approval=[-2.5cm][1]{aprovalsheet},
  abstract={Resumo}{abstract_ptbr},
  abstract=[english]{Abstract}{abstract_en},
  % abstract={Resumo Estendido}{resumoest},
  % dedication={dedication},
  % ack=[Acknowledgments]{thanks},
% ack=[Acknowledgments]{ack},
  epigraphtext={A ciência pode ser descrita como a arte da simplificação sistemática.}{Karl Popper}
}

\chapter{Introdução} % (fold)
\label{cha:introdu_o}
% !TEX root = pfc.tex

O tráfego de veículos representa um fenômeno de grande importância socioeconômica, principalmente nos grandes centros urbanos cujos deslocamentos, no menor tempo possível, são uma necessidade cotidiana. Projetar sistemas viários que absorvam toda a frota veicular desses grandes centros e minimizam os congestionamentos exige ferramentas cujo desenvolvimento ainda representa objeto de estudo para diversos grupos de pesquisadores.

Vários fatores interferem na qualidade do tráfego, sendo o principal o crescente número de veículos nos centros urbanos. Segundo \cite{fenabrave:2013:online} - Federação Nacional da Distribuição de Veículos Automotores, as vendas de veículos cresceram cerca de 381\% entre janeiro de 2003 e janeiro de 2011. A facilitação ao crédito pessoal, os longos financiamentos oferecidos pelas concessionárias e a redução do IPI (Imposto sobre Produtos Industrializados) incentivaram o consumo de carros no Brasil, explicando essa tendência ascendente no número de automóveis.

% \begin{figure}[tb]
%   \begin{center}
%     \includegraphics[scale=0.8]{imgs/emplacamentos.png}
%   \end{center}
%   \caption{Crescimento mensal no número de emplacamentos de veículos comerciais leves no período 1997 a 2011 \citep{fenabrave:2013:online}.} 
%   \label{fig:fenabreve}
% \end{figure}

A falta de planejamento urbano e o crescente aumento de veículos, a péssima qualidade do transporte público no Brasil e o excessivo número de acidentes de trânsito são umas das principais causas de congestionamentos. Nos últimos anos, milhões de pessoas perderam tempo e dinheiro devido a problemas relacionados ao trânsito \citep{pamela:2012:masther}. Diante desse fato, a administração pública deve priorizar o assunto mobilidade, como tem sido feito nas principais regiões metropolitanas brasileiras. Através de pesquisas pode-se levantar dados que contribuem para análise e simulação do tráfego. Informações como fluxo de veículos, volume de tráfego e matriz O/D (origem/destino) são de grande relevância no estudo dessa área, formando uma base teórica sólida para ser utilizada na resolução de problemas dessa natureza.

Na área de simulação do tráfego existem várias empresas atuantes: \cite{vissim:2013:online} é atualmente a líder mundial de mercado em simulação de fluxo de tráfego multi-modal de microregiões. Outras companhias, \cite{arcady:2013:online} e \cite{citilabs:2013:online}, também fornecem produtos com capacidades de predição, simulação de congestionamentos, atrasos, acidentes, além de gerar relatórios detalhados e animações 2D e 3D. SUMO - \textit{Simulation of Urban MObility} é uma alternativa \textit{open source}\footnote{O termo código aberto, ou \textit{open source} em inglês, foi criado pela OSI e refere-se a \textit{software} livre. Mais informações em \url{http://opensource.org}} e gratuita para simulação do tráfego, diferente dos produtos proprietários e de alto custo citados anteriormente \citep{SUMO2011}.

Segundo \cite{dnit:2011:online}, o PNCT - Plano Nacional de Contagem de Trânsito vem armazenando nos últimos anos importantes dados de volume de tráfego e já possui uma significativa série histórica desses dados. A informação coletada é de grande importância pois seus resultados são subsídios básicos para estudos econômicos, projetos rodoviários e planejamento de tráfego, além do papel essencial no estabelecimento de critérios para o cumprimento das seguintes finalidades:

\begin{itemize}
  \item Planejar o sistema rodoviário;
  \item Programar necessidades e prioridades de melhorias no sistema rodoviário;
  \item Estabelecer as tendências de tráfego no futuro;
  \item Avaliar o fluxo existente de tráfego em relação ao sistema rodoviário atual;
  \item Justificar e planejar o policiamento;
  \item Estudos de localização de postos de pesagem, socorro médico emergencial, etc.;
  \item Projetar pavimento, obras de arte, seção transversal e outros elementos de rodovia \citep{dnit:2011:online}. 
\end{itemize}

Existem métodos manuais e automáticos para realização das contagens volumétrica e classificatória, que normalmente são de alto custo financeiro. De acordo com a forma de instalação dos equipamentos de contagem, tais métodos podem ser invasivos ou não-invasivos. Os métodos invasivos necessitam de instalações junto ou sob a camada asfáltica. Já os métodos não-invasivos normalmente utilizam câmeras, sensores ou instalações sobre o solo \citep{goldner:2009:misc}.

Este trabalho tem como propósito desenvolver um método computacional de simples implementação, configuração, instalação e operação, capaz de realizar a contagem volumétrica de veículos de forma não-invasiva e que possa ser utilizado como uma alternativa de baixo custo financeiro, se comparado aos sitemas de contagem volumétrica convencionais.

\section{Objetivos} % (fold)
\label{sec:objetivos}

O objetivo principal do trabalho é desenvolver um sistema de visão computacional que auxilie na análise das condições do tráfego urbano. São os objetivos específicos deste estudo:
\begin{itemize}
  \item Entender a utilização de um sistema de visão para problemas de transporte;
  \item Realizar a contagem volumétrica dos veículos através de um método não-invasivo de simples implementação;
  \item Analisar as condições do tráfego urbano através de imagens coletadas por uma câmera digital.
\end{itemize}

% section objetivos (end)

\section{Apresentação da Empresa} % (fold)
\label{sec:apresenta_o_da_empresa}

TODO: não é mais empresa

% A NunesCV é uma startup que foi criada no início de 2013 e desenvolve soluções e produtos de tecnologia para o setor de serviços. A empresa desenvolve sistemas baseados em visão computacional, como um produto para leitura automática de cartões de ponto utilizando tecnologia própria de Reconhecimento Ótico de Caracteres (OCR).

% Outro tipo de aplicação desenvolvida pela empresa é um sistema de gerenciamento remoto de uma rede de mídia indoor, que são televisões instaladas em pontos comerciais estratégicos anunciando informações úteis para os usuários. Cada ponto pode ter o conteúdo atualizado em tempo real via Internet, além de informar se a televisão está ligada e se o conteúdo está correto. A empresa desenvolve o software para um sistema embutido que controla o aparelho, o sistema web de gerenciamento e atualização de conteúdo e especifica o hardware necessário para o funcionamento do sistema.

% section apresenta_o_da_empresa (end)

\section{Organização do Texto} % (fold)
\label{sec:organiza_o_do_texto}

A monografia está dividida em seis capítulos:

\begin{itemize}
  \item Capítulo \ref{cha:introdu_o} é introdutório e são apresentados alguns pontos que motivam e justificam o desenvolvimento do projeto, além de uma listagem dos objetivos específicos que definem o escopo do mesmo.
  \item Capítulo \ref{cha:contagem_de_ve_culos_e_monitoramento_do_tr_fego} apresenta as tecnologias de contagem volumétrica e classificatória existentes. É também apresentado uma pesquisa bibliográfica sobre sistemas de visão computacional inseridos no contexto da Engenharia de Transporte.
  \item Capítulo \ref{cha:vis_o_computacional} introduz a teoria de visão computacional, contextualizando o leitor. Trata-se de uma apresentação da tecnologia com foco em conceitos e técnicas de processamento de imagens.
  \item Capítulo \ref{cha:metodologia} apresenta a metodologia para contagem volumétrica automática de veículos. As técnicas utilizadas, a sequência de operações, as imagens de processamento, as condições de captura, a estrutura do software e os métodos para análise de resultados são mostrados e explicados. 
  \item Capítulo \ref{cha:testes_e_resultados} apresenta os resultados e a validação das contagens; uma análise crítica do método proposto mostrando como as características da cena influenciam no resultado. 
  \item Capítulo \ref{cha:considera_es_finais} são feitas as conclusões e as considerações finais do trabalho, bem como recomendações para trabalhos futuros.
\end{itemize}


% section organiza_o_do_texto (end)
% chapter introdu_o (end)

\chapter{Contagem de Veículos e Monitoramento do Tráfego} % (fold)
\label{cha:contagem_de_ve_culos_e_monitoramento_do_tr_fego}
% !TEX root = pfc.tex

\section{Equipamentos invasivos} % (fold)
\label{sec:equipamentos_invasivos}

\subsection{Tubo pneum�tico} % (fold)
\label{sub:tubo_pneum_tico}

% subsection tubo_pneum_tico (end)

\subsection{Detector de la�o indutivo} % (fold)
\label{sub:detectores_de_la_os_indutivos}

% subsection detectores_de_la_os_indutivos (end)

\subsection{Sensor magn�tico} % (fold)
\label{sub:sensores_magn_ticos}

% subsection sensores_magn_ticos (end)

\subsection{Sensor piezoel�trico} % (fold)
\label{sub:sensores_piezoel_tricos}

% subsection sensores_piezoel_tricos (end)

% section equipamentos_invasivos (end)

\section{Equipamentos n�o-invasivos} % (fold)
\label{sec:equipamentos_n_o_invasivos}

\subsection{C�mera de v�deo} % (fold)
\label{sub:c_mera_de_v_deo}

% subsection c_mera_de_v_deo (end)

\subsection{Radar por microondas} % (fold)
\label{sub:radar_por_microondas}

% subsection radar_por_microondas (end)

\subsection{Sensor infravermelho} % (fold)
\label{sub:sensor_infravermelho}

% subsection sensor_infravermelho (end)

\subsection{Sensor de vetor de ac�stica passiva} % (fold)
\label{sub:sensor_de_vetor_de_ac_stica_passiva}

% subsection sensor_de_vetor_de_ac_stica_passiva (end)

% section equipamentos_n_o_invasivos (end)

\section{Aplica��es de Vis�o Computacional} % (fold)
\label{sec:aplica_es_de_vis_o_computacional}

\subsection{Dire��o aut�noma de ve�culo inteligente} % (fold)
\label{sub:dire_o_aut_noma_de_ve_culo_inteligente}

\cite{thrun:2006:article}

% subsection dire_o_aut_noma_de_ve_culo_inteligente (end)

\subsection{Detec��o de placa de ve�culos} % (fold)
\label{sub:detec_o_de_placa_de_ve_culos}

\cite{martinsky:2007:masther}

% subsection detec_o_de_placa_de_ve_culos (end)

\subsection{Contagem volum�trica utilizando dispositivos m�veis} % (fold)
\label{sub:contagem_volum_trica_utilizando_dispositivos_m_veis}

\cite{feitosa:2012:masther}

% subsection contagem_volum_trica_utilizando_dispositivos_m_veis (end)

% section aplica_es_de_vis_o_computacional (end)
% chapter contagem_de_ve_culos_e_monitoramento_do_tr_fego (end)

\chapter{Visão Computacional} % (fold)
\label{cha:vis_o_computacional}
% !TEX root = pfc.tex

Segundo \cite{linda:2001:book}, vis�o computacional � um campo da Ci�ncia da Computa��o que inclui m�todos para a aquisi��o, processamento e an�lise de imagens. \cite{morris:2004:book} define processamento de imagem e vis�o computacional separadamente. Para \citeauthor{morris:2004:book}, processamento de imagem � a manipula��o de uma imagem digital, gerando como resultado uma nova representa��o da mesma imagem, ao passo que vis�o computacional � a extra��o de informa��es num�ricas ou simb�licas de imagens. Em linhas gerais, vis�o computacional � a tecnologia que transforma imagens do mundo real em uma representa��o que computadores s�o capazes de interpretar e processar e, com isso, produzir informa��es �teis para aplica��es de Engenharia. Contempla uma base te�rica e tecnol�gica para a constru��o de sistemas artificiais que obt�m informa��es de imagens ou quaisquer dados multidimensionais. Mas como os computadores podem entender o mundo visual dos humanos? Quais vantagens e quais tipos de aplica��es podem existir a partir desse conceito?

Segundo \cite{szeliski:2010:book}, o ser humano � perfeitamente capaz de perceber a estrutura tridimensional do mundo ao seu redor. Ao visualizar a Figura \ref{fig:flower}, percebe-se que a vis�o humana interpreta varia��es de transpar�ncia e sombra, al�m de diferenciar o objeto do \textit{backgroud}\footnote{Segundo plano ou plano de fundo} com facilidade. Isso acontece porque o c�rebro humano divide o sinal de vis�o em muitos canais, gerando um fluxo de diferentes tipos de informa��o. Ele � capaz de identificar quais s�o as partes importantes de uma imagem a serem examinadas e ao mesmo tempo suprimir a aten��o para regi�es menos importantes \citep{szeliski:2010:book}. Al�m disso, o c�rebro possui um sistema de realimenta��o poderoso que implementa um controle em malha fechada, composto por sensores (vis�o, audi��o, olfato, tato e paladar) e atuadores (�ris para controlar a entrada de luminosidade nos olhos) \citep{opencv:2008:book}.

\begin{figure}[ht]
  \begin{center}
    \includegraphics{imgs/flower.png}
  \end{center}
  \caption{O ser humano � capaz de determinar a forma e a transpar�ncia de cada p�tala de uma flor \citep{szeliski:2010:book}.}
  \label{fig:flower}
\end{figure}

Diante da facilidade com que o ser humano enxerga o mundo ao seu redor � intuitivo pensar que o processamento de imagens por vis�o computacional � simples. Mas isso n�o � verdade. Em um sistema de vis�o, o computador recebe, na maioria das vezes, apenas uma matriz de n�meros inteiros, em que cada posi��o � denominada pixel, como mostrado na Figura \ref{fig:opencv_car}. Todos aqueles padr�es de interpreta��o de informa��o presentes no c�rebro n�o existem aqui. Al�m disso, deve-se considerar os ru�dos existentes num sistema de vis�o, que diminuem ainda mais a quantidade de informa��o dos dados. Esse tipo de problema pode ser causado devido a varia��es no ambiente (luminosidade, clima, reflexos, movimenta��es), imperfei��es na captura de imagem (lente, configura��o mec�nica), ru�dos el�tricos no sensor �ptico da c�mera e compress�o das imagens ap�s a captura \citep{opencv:2008:book}.

\begin{figure}[ht]
  \begin{center}
    \includegraphics{imgs/opencv_car.png}
  \end{center}
  \caption{Para um computador, o retrovisor de um carro � apenas uma matriz composta por pixeis \citep{opencv:2008:book}.}
  \label{fig:opencv_car}
\end{figure}

Mesmo com todos esses desafios, por incr�vel que pare�a, � poss�vel desenvolver sistemas baseados em vis�o computacional bastante robustos e com alto grau de confiabilidade. Isso come�a a se tornar poss�vel quando as imagens capturadas est�o inseridas no contexto de uma determinada aplica��o. Por exemplo: se um sistema de vis�o computacional tem por objetivo rastrear carros, n�o faz nenhum sentindo buscar os ve�culos em �reas que n�o sejam as ruas, avenidas e rodovias. Caso um objeto seja encontrado em uma �rea verde ou no azul do c�u, a probabilidade de que esse objeto seja um carro � muito baixa. � uma an�lise �bvia mas de suma import�ncia no processamento de imagens.

O uso de m�todos estat�sticos em vis�o computacional tamb�m s�o primordiais, indo de encontro ao problema de ru�dos discutido anteriormente. Considerar a m�dia dos pixeis no tempo � uma abordagem estat�stica, que pode vir a ser uma solu��o para problemas que envolvem imagens ruidosas. Outra t�cnica bastante comum � a constru��o de modelos das c�meras, capazes de caracteriz�-las matematicamente atrav�s de seus par�metros intr�nsecos e extr�nsecos. Os par�metros internos da c�mera como dist�ncia focal, distor��es de lente e tamanho do pixel correspondem aos par�metros intr�nsecos. Os par�metros extr�nsecos est�o relacionados � orienta��o e posi��o da c�mera em rela��o a um sistema de refer�ncia no mundo. Com esses par�metros fica simples corrigir imperfei��es nas imagens, que podem ocorrer devido a lente ou alguma configura��o mec�nica \citep{opencv:2008:book}.

Essas e mais uma s�rie de t�cnicas, descritas nas se��es seguintes, s�o objetos de estudo no campo de vis�o computacional. Elas em conjunto s�o combinadas e organizadas de maneira a transformar quaisquer dados multidimensionais em informa��es de alto n�vel, como por exemplo aus�ncia e presen�a de um componente, dimens�o e cor de objetos. Como regra geral: quanto mais restrito � o escopo de uma aplica��o de vis�o computacional, mais o problema pode ser simplificado e mais confi�vel ser� a solu��o final \citep{opencv:2008:book}.

A seguir � realizado uma breve revis�o bibliogr�fica sobre conceitos, t�cnicas e m�todos de processamento de imagens digitais aplicados ao campo da vis�o computacional.

\section{Imagem digital} % (fold)
\label{sec:imagem_digital}

De acordo com \cite{woods:2000:book}, uma imagem pode ser definida como uma fun��o bidimensional de intensidade de luz $f(x,y)$, em que $x$ e $y$ s�o cordenadas espaciais de um plano, e a amplitude de $f$ � proporcional ao brilho (ou n�veis de cinza) da imagem naquele ponto. A Figura \ref{fig:imagem_digital} ilustra a conven��o dos eixos adotada nesse e na maioria dos trabalhos da �rea.

\begin{figure}[ht]
  \begin{center}
    \includegraphics[scale=0.7]{imgs/imagem_digital.png}
  \end{center}
  \caption{Conven��o dos eixos para representa��o de imagens digitais \citep{woods:2000:book}.}
  \label{fig:imagem_digital}
\end{figure}

Uma imagem digital � a discretiza��o de uma imagem $f(x,y)$, tanto em coordenadas espaciais quanto em brilho. A OpenCV \citep{opencv_library}, uma biblioteca de vis�o computacional descrita na Se��o \ref{sec:biblioteca_opencv}, considera uma imagem digital como sendo uma matriz cujos �ndices de linhas e colunas determinam um ponto na imagem, e o valor correspondente do elemento da matriz identifica o n�vel de cinza naquele ponto. Os elementos dessa matriz digital s�o chamados de pixels, embora existam outros nomes como elementos da imagem, elementos da figura ou pels \citep{woods:2000:book}.

% section imagem_digital (end)

\section{Modelo RGB de cores} % (fold)
\label{sec:modelo_rgb_de_cores}

Segundo \cite{woods:2000:book}, � comprovado que o olho humano possui entre 6 e 7 milh�es de cones\footnote{C�lulas fotorreceptoras localizadas na retina do olho que s�o respons�veis pela vis�o das cores.}, que podem ser dividos em tr�s principais categorias de sensoriamento, aproximadamente correspondentes �s cores vermelho, verde e azul. Desse total de cones existentes, cerca de 65\% s�o sens�veis � luz vermelha, 33\% s�o sens�veis � luz verde e apenas 2\% deles s�o sens�veis � luz azul, sendo os cones azuis os que possuem maior grau de sensibilidade � luz. Dessa forma, as cores s�o vistas pelo olho humano como combina��es das chamadas cores prim�rias: vermelho (R, \textit{red}), verde (G, \textit{green}) e azul (B, \textit{blue}), fato que inspirou estudos do modelo RGB de cores.

O modelo RGB � um sistema de coordenadas cartesianas

\begin{figure}[ht]
  \begin{center}
    \begin{subfigure}[b]{.49\textwidth}
      \begin{center}
        \includegraphics[width=.7\linewidth]{imgs/cubo_rgb.png}
      \end{center}
      \caption{}
      \label{fig:cubo_rgb}
    \end{subfigure}
    \begin{subfigure}[b]{.49\textwidth}
      \begin{center}
        \includegraphics[width=.7\linewidth]{imgs/colorcube.png}
      \end{center}
      \caption{}
      \label{fig:colorcube}
    \end{subfigure}
  \end{center}
  \caption{(a) Cubo de cores RGB. Os pontos ao longo da diagonal principal t�m valores da escala de cinza que v�o do preto $(0,0,0)$ ao branco $(255,255,255)$ \citep{woods:2000:book}; (b) Representa��o em cores do cubo RGB \citep{theo:2003:online}.}
  \label{fig:cubos}
\end{figure}

% section modelo_rgb_de_cores (end)

\section{Escala de cinza} % (fold)
\label{sec:escala_de_cinza}

% section escala_de_cinza (end)

\section{Filtros} % (fold)
\label{sec:filtros}

% section filtros (end)

\section{Limiariza��o} % (fold)
\label{sec:limiariza_o}

% section limiariza_o (end)

\section{Subtra��o de fundo} % (fold)
\label{sec:subtra_o_de_background}

% section subtra_o_de_background (end)

\section{Opera��es morfol�gicas} % (fold)
\label{sec:opera_es_morfol_gicas}

\subsection{Eros�o} % (fold)
\label{sub:eros_o}

\subsection{Dilata��o} % (fold)
\label{sub:dilata_o}

\subsection{Abertura e fechamento} % (fold)
\label{sub:abertura_e_fechamento}

% subsection abertura_e_fechamento (end)

% subsection dilata_o (end)

% subsection eros_o (end)

% section opera_es_morfol_gicas (end)

\section{Detec��o de caracter�sticas relevantes} % (fold)
\label{sec:detec_o_de_caracter_sticas_relevantes}

% section detec_o_de_caracter_sticas_relevantes (end)

\section{Biblioteca OpenCV} % (fold)
\label{sec:biblioteca_opencv}

OpenCV (\textit{Open Source Computer Vision}) \citep{opencv_library} � uma biblioteca de vis�o computacional \textit{open source}\footnote{O termo c�digo aberto, ou \textit{open source} em ingl�s, foi criado pela OSI e refere-se a \textit{software} livre. Mais informa��es em \url{http://opensource.org}} dispon�vel em: \url{http://sourceforge.net/projects/opencvlibrary/}. O c�digo � escrito em C e C++ e pode ser compilado e executado em ambientes Linux, Windows e Mac OS X. Existem tamb�m interfaces em Python, Ruby, Matlab e outras linguagens. Cont�m mais de 500 algoritmos otimizados para processamento de imagens e v�deos, cobrindo diversas �reas da vis�o computacional, como: inspe��o industrial, imagens m�dicas, seguran�a, interface com o usu�rio, calibra��o de c�mera, vis�o est�reo, rob�tica e \textit{machine learning} \citep{opencv:2008:book}.

Segundo \cite{opencv2:2011:book}, desde o lan�amento da OpenCV em 1999, ela vem sendo largamente adotada como a principal ferramenta de desenvolvimento pela comunidade de pesquisadores e programadores em vis�o computacional. A biblioteca foi originalmente desenvolvida pela Intel \citep{intel:2013:online}, por um time liderado por Gary Bradski e com o prop�sito de avan�ar em pesquisas na �rea de vis�o. Depois de uma s�rie de vers�es \textit{Beta}, a vers�o 1.0 foi lan�ada em 2006. O segundo maior lan�amento aconteceu em 2009 com a OpenCV 2, propondo importantes modifica��es em sua estrutura e especialmente a nova interface C++. Atualmente a OpenCV encontra-se na vers�o 2.4.

\cite{opencv:2008:book} afirmam que a licen�a \textit{open source} da OpenCV permite que aplica��es comerciais possam ser construidas utilizando parte ou toda a biblioteca, sem a necessidade de que o c�digo da aplica��o seja aberto. Devido a essa pol�tica liberal de uso, gigantes como IBM, Microsoft, Intel, SONY, Siemens, Google e outros, al�m dos centros de pesquisa Stanford, MIT, CMU, Cambridge, INRIA e outros, utilizam a biblioteca em seus projetos e pesquisas. A OpenCV foi pe�a chave no sistema de vis�o de um rob� desenvolvido em Stanford, conhecido como Stanley, que ganhou a corrida de carros aut�nomos \$2M DARPA Grand Challenge \citep{thrun:2006:article}. Um relato desse caso � feito na Subse��o \ref{sub:dire_o_aut_noma_de_ve_culo_inteligente}.

% section biblioteca_opencv (end)

\section{Aplica��es existentes} % (fold)
\label{sec:aplica_es_existentes}

\cite{ivision:2013:online} possui diversas aplica��es concebidas utilizando os conhecimentos de vis�o computacional. O fato dessa tecnologia ser n�o intrusiva, ou seja, n�o altera em nada o meio em que est� sendo utilizada, torna os sistemas de vis�o realiz�veis em grande parte dos processos industriais, urbanos e ambientais.

As aplica��es em an�lise dimensional se caracterizam por efetuarem medidas em objetos, sendo elas lineares e/ou angulares. A an�lise dimensional por imagem � vantajosa pois possibilita que as medidas sejam feitas � dist�ncia, quando n�o � poss�vel ou desej�vel o contato direto com o objeto. C�meras de alta resolu��o se aplicam nesses projetos, garantindo medi��es precisas e com repetibilidade. Muito comum na siderurgia, esse tipo de aplica��o possibilita a medi��o em objetos que se encontram em altas temperaturas. As c�meras infravermelho tem sido utilizadas nesse tipo de aplica��o.

O uso de reconhecimento de padr�es permite a identifica��o de caracter�sticas de um produto comparando-o com um modelo predeterminado. Essas caracter�sticas a serem inspecionadas s�o escolhidas de modo a identificar e diferenciar um tipo de modelo de outro. Assim � poss�vel dizer se um objeto est� conforme um padr�o ou n�o.

Inspe��o de n�vel � bastante comum na ind�stria de bebidas e na ind�stria farmac�utica para inspe��o de enchimento de ampolas, vidros de medicamentos ou qualquer recipiente transl�cido que contenha l�quido.

A inspe��o por an�lise de cores possibilita a cria��o de solu��es para separar produtos por cor ou verificar se a tonalidade est� igual a uma amostra. Tem a vantagem de ser um m�todo determin�stico em rela��o a uma inspe��o de cores realizada por seres humanos, onde pode existir subjetividade. No entanto, varia��es de luminosidade podem dificultar a realiza��o desse tipo de inspe��o.

Existem tamb�m aplica��es de rastreabilidade que envolvem leituras de c�digos de barras ou de c�digos bidimensionais como \textit{Data Matrix}\footnote{C�digo de barras matricial bi-dimensional que consiste em c�lulas brancas ou pretas arranjadas em forma de quadrado ou ret�ngulo. Se caracteriza pelas bordas em formato de L. Armazena no m�ximo 2335 caracteres alfanum�ricos.} e o \textit{QR Code}\footnote{\textit{Quick Response Code} � um c�digo de barras bi-dimensional criado pela empresa japonesa Denso-Wave em 1994. Pode armazenar at� 7089 caracteres, dependendo do tipo de dado. Possui redund�ncia em sua codifica��o, possibilitando corre��o e recupera��o de informa��o.}. Permitem identificar e rastrear os produtos, bem como armazenar grandes quantidades de dado. S�o sistemas frequentemente utilizados na ind�stria automobil�stica mas s�o aplic�veis em qualquer projeto que necessite de controle de produ��o.

As aplica��es para contagem, sele��o e classifica��o podem instrumentar diversas ind�strias, como por exemplo a ind�stria farmac�utica para contagem de medicamentos. Tamb�m � comum na produ��o de componentes eletr�nicos para verifica��o de pinos e conectores. Na engenharia de transporte podem contribuir na an�lise do tr�fego realizando contagem volum�trica de ve�culos e pessoas.
% chapter vis_o_computacional (end)

\chapter{Metodologia} % (fold)
\label{cha:metodologia}
% !TEX root = pfc.tex


Como visto no Cap�tulo \ref{cha:vis_o_computacional}, existem diversas t�cnicas de processamento digital de imagens que possibilitam que aplica��es de vis�o computacional sejam capazes de extrair informa��es a partir de uma ou mais imagens de entrada.

\begin{figure}[ht]
  \begin{center}
    \begin{subfigure}[b]{.49\textwidth}
      \begin{center}
        \includegraphics[width=1\linewidth]{imgs/cena_captura.png}
      \end{center}
      \caption{}
      \label{fig:cena_captura}
    \end{subfigure}
    \begin{subfigure}[b]{.49\textwidth}
      \begin{center}
        \includegraphics[width=1\linewidth]{imgs/original_frame.png}
      \end{center}
      \caption{}
      \label{fig:original_frame}
    \end{subfigure}
  \end{center}
  \caption{(a) Posicinamento da c�mera para captura de imagens. A �rea em destaque simboliza }
  \label{fig:cena}
\end{figure}


\begin{figure}[ht]
  \begin{center}
    \includegraphics[scale=0.9]{imgs/general_process.pdf}
  \end{center}
  \caption{Fluxograma com a representa��o global do m�todo de contagem.}
  \label{fig:general_process}
\end{figure}

\section{Entrada de dados} % (fold)
\label{sec:entrada_de_dados}

\begin{lstlisting}
#include <iostream>

#include <opencv2/core/core.hpp>
#include <opencv2/highgui/highgui.hpp>
#include <opencv2/imgproc/imgproc.hpp>

int main(int argc, char const *argv[])
{
  cv::VideoCapture capture;
  if(!capture.isOpened()) {
    std::cout << "can not open camera or video file" << std::endl;
    return;
  }
  ...  
  for(;;) {
    cv::Mat frame;
    capture >> frame;

    if(frame.empty()) break; 
    ...  
  }

  return 0;
}
\end{lstlisting}

\begin{figure}[ht]
  \begin{center}
    \includegraphics[scale=0.5]{imgs/frame.png}
  \end{center}
  \caption{\textit{Frame} obtido a partir do v�deo de entrada.}
  \label{fig:frame_in}
\end{figure}

% section entrada_de_dados (end)

\section{Pr�-processamento} % (fold)
\label{sec:pr_processamento}

\begin{lstlisting}
  ...
  for(;;) {
    ...
    cv::Mat gray;
    cv::cvtColor(frame, gray, CV_BGR2GRAY);

    cv::GaussianBlur(gray, gray, cv::Size(7, 7), 3);
    ...
  }
  ...  
\end{lstlisting}

\begin{figure}[ht]
  \begin{center}
    \includegraphics[scale=0.5]{imgs/gray.png}
  \end{center}
  \caption{Resultado da etapa de pr�-processamento, uma imagem filtrada em escala de cinza.}
  \label{fig:pre_processamento}
\end{figure}

% section pr_processamento (end)

\section{Subtra��o de \textit{background}} % (fold)
\label{sec:subtra_o_de_background}

\begin{lstlisting}
  ...
  cv::BackgroundSubtractorMOG2 model;
  for(;;) {
    ...
    cv::Mat foreground;
    model(gray, foreground);
    ...
  }
  ...
\end{lstlisting}

\begin{figure}[ht]
  \begin{center}
    \includegraphics[scale=0.5]{imgs/foreground.png}
  \end{center}
  \caption{Detec��o do \textit{foreground} na etapa de subtra��o de fundo.}
  \label{fig:foreground}
\end{figure}

% section subtra_o_de_background (end)

\section{Binariza��o} % (fold)
\label{sec:binariza_o}

\begin{lstlisting}

void bin(cv::Mat &src)
{
    //conta os pixeis cinza
    int sumGray = 0, sumWhite = 0;
    for(int i = 0; i < src.rows; i++) {
        const uchar* ptri = src.ptr<uchar>(i);
        for(int j = 0; j < src.cols; j++) {
            if(ptri[j] != 0 && ptri[j] != 255)
                sumGray++;
            else if(ptri[j] == 255)
                sumWhite++;
        }
    }

    if(sumWhite == 0) {
        src.setTo(cv::Scalar(0));
    }
    else if((float)sumGray/(float)sumWhite > 10.0)
        cv::threshold(src, src, 250, 255, CV_THRESH_BINARY);
    else
        cv::threshold(src, src, 5, 255, CV_THRESH_BINARY);
}

  ...
  for(;;) {
    ...
    bin(foreground);

    cv::Mat morph;
    cv::Mat element = cv::getStructuringElement(cv::MORPH_ELLIPSE,
                                                cv::Size(5,5));
    cv::morphologyEx(foreground, morph, CV_MOP_CLOSE, element, 
                     cv::Point(-1,-1), 3);
    ...
  }
  ...  
\end{lstlisting}

\begin{figure}[ht]
  \begin{center}
    \begin{subfigure}[b]{.49\textwidth}
      \begin{center}
        \includegraphics[width=1\linewidth]{imgs/bin.png}
      \end{center}
      \caption{}
      \label{fig:bin}
    \end{subfigure}
    \begin{subfigure}[b]{.49\textwidth}
      \begin{center}
        \includegraphics[width=1\linewidth]{imgs/morph.png}
      \end{center}
      \caption{}
      \label{fig:morph}
    \end{subfigure}
  \end{center}
  \caption{Resultado da etapa de binariza��o. (a) Imagem binarizada utilizando um limiar simples; (b) Opera��o morfol�gica de fechamento.}
  \label{fig:bin_morph}
\end{figure}

% section binariza_o (end)

\section{Detec��o de \textit{blobs}} % (fold)
\label{sec:detec_o_de_blobs}

\begin{lstlisting}
  ...
    cv::SimpleBlobDetector::Params params;
    params.filterByInertia = false;
    params.filterByConvexity = false;
    params.filterByColor = true;
    params.blobColor = 255;
    params.filterByCircularity = false;
    params.filterByArea = true;
    params.minArea = 500.0f;
    params.maxArea = 80000.0f;

    cv::Ptr<cv::FeatureDetector> detector = 
                                 new cv::SimpleBlobDetector(params);
    detector->create("SimpleBlob");
    ...
    for(;;) {
      ...
      std::vector<cv::KeyPoint> keypoints;
      detector.detect(morph, keypoints)
    }
    ...  
\end{lstlisting}

\begin{figure}[ht]
  \begin{center}
    \includegraphics[scale=0.5]{imgs/keypoints.png}
  \end{center}
  \caption{Resultado da etapa de detec��o de \textit{blobs}.}
  \label{fig:keypoints}
\end{figure}

% section detec_o_de_blobs (end)

\section{Rastreamento e contagem} % (fold)
\label{sec:rastreamento_e_contagem}

\begin{figure}[ht]
  \begin{center}
    \includegraphics[scale=0.85]{imgs/fluxograma_contagem.pdf}
  \end{center}
  \caption{Fluxograma do algoritmo de rastreamento e contagem de ve�culos.}
  \label{fig:fluxograma_contagem}
\end{figure}

\begin{figure}[ht]
  \begin{center}
    \includegraphics[scale=0.5]{imgs/trackers.png}
  \end{center}
  \caption{Resultado da etapa de rastreamento e contagem, ilustrando \textit{trackers}, \textit{keypoints} e a regi�o de contagem.}
  \label{fig:trackers}
\end{figure}

% section rastreamento_e_contagem (end)
% chapter metodologia (end)

\chapter{Testes e Resultados} % (fold)
\label{cha:testes_e_resultados}
% !TEX root = pfc.tex

O m�todo proposto foi implementado em forma de um \textit{software} multiplataforma, desenvolvido em linguagem \verb!C++! \citep{cplusplus:2013:online}, com intuito de obter um bom desempenho de execu��o e f�cil implementa��o utilizando os conceitos de programa��o orientada a objetos.

Como biblioteca para processamento das imagens, foi utilizado a OpenCV 2.4.4 \citep{opencv_library}, contribuindo para que opera��es complexas de vis�o computacional pudessem ser realizadas com poucas linhas de c�digo. O desenvolvimento do m�todo n�o est� preso � tecnologia escolhida, podendo ser empregado utilizando outras linguagens de programa��o, bem como outras bibliotecas. A tecnologia foi escolhida visando a simplicidade no processo de desenvolvimento das opera��es de processamento de imagens, que est�o bem detalhadas pelos trechos de c�digo na Se��o \ref{sec:fluxo_de_processos}.

\section{Testes} % (fold)
\label{sec:testes}

Todos os v�deos foram capturados por um aparelho celular Samsung OMNIA W GT-I8350 \citep{omnia:2013:online}, como descrito na Se��o \ref{sec:caracter_sticas_de_captura}. As imagens foram feitas na passarela localizada na Av. Presidente Carlos Luz, nos dois sentidos da via, como ilustra a Figura \ref{fig:sentidos}. Duas configura��es de captura foram utilizadas:

\begin{itemize}
  \item HD 720p: gerando v�deos com resolu��o $1280\times 720$, e \textit{framerate} de 29 FPS;
  \item VGA: gerando v�deos com resolu��o $640\times 480$, e \textit{framerate} de 29 FPS.
\end{itemize}

\noindent Uma terceira resolu��o foi obtida a partir dos v�deos em HD, aplicando uma opera��o para redimension�-los com altura e largura a metade do valor original. Essas varia��es de cena e resolu��o foram feitas com o objetivo de avaliar o comportamento do m�todo em diferentes condi��es. Assim, ser� poss�vel identificar o impacto de cada vari�vel no resultado final. A Tabela \ref{tab:videos_teste} lista as amostras utilizadas.

\begin{figure}[ht]
  \begin{center}
    \begin{subfigure}[b]{.49\textwidth}
      \begin{center}
        \includegraphics[width=1\linewidth]{imgs/trackers.png}
      \end{center}
      \caption{}
      \label{fig:sentido_pampulha}
    \end{subfigure}
    \begin{subfigure}[b]{.49\textwidth}
      \begin{center}
        \includegraphics[width=1\linewidth]{imgs/sentido_centro.png}
      \end{center}
      \caption{}
      \label{fig:sentido_centro}
    \end{subfigure}
  \end{center}
  \caption{Resultado do m�todo de contagem para dois tipos de cena. (a) Av. Carlos Luz, sentido Pampulha; (b) Av. Carlos Luz, sentido Centro.}
  \label{fig:sentidos}
\end{figure}

\begin{table}[ht]
  \caption{V�deos utilizados nos testes.}
  \label{tab:videos_teste}
  \begin{center}
    \begin{tabular}{lcccc}
    \toprule
    \textbf{Nome do v�deo} & \textbf{Formato} & \textbf{Resolu��o} & \textbf{Dura��o} & \textbf{FPS} \\
    \midrule
      carlos\_luz\_centro\_vga & mp4 & $ 640\times 480 $ & 00:05:23 & 29 \\
      carlos\_luz\_centro\_hd\_resized & mpeg & $ 640\times 360 $ & 00:05:25 & 29 \\
      carlos\_luz\_centro\_hd & mp4 & $ 1280\times 720 $ & 00:05:29 & 29 \\
      carlos\_luz\_pampulha\_vga\_1 & mp4 & $ 640\times 480 $ & 00:05:43 & 29 \\
      carlos\_luz\_pampulha\_vga\_2 & mp4 & $ 640\times 480 $ & 00:05:19 & 29 \\
      carlos\_luz\_pampulha\_hd\_2\_resized & mpeg & $ 640\times 360 $ & 00:06:19 & 29 \\
      carlos\_luz\_pampulha\_hd\_2 & mp4 & $ 1280\times 720 $ & 00:06:21 & 29 \\
    \bottomrule
    \end{tabular}
  \end{center}
\end{table}

Tanto o desenvolvimento do \textit{software}, quanto os testes, foram feitos em um \textit{Laptop} Dell Vostro V131, Processador Intel Core I3-2330M (2.20GHZ, dual core 4 Threads, 3MB L3 cache), 8GB de mem�ria RAM, em ambiente Linux Fedora 18 (Spherical Cow).

% section testes (end)

\section{Matriz de confus�o} % (fold)
\label{sec:matriz_de_confus_o}

A matriz de confus�o cont�m informa��es sobre classifica��es reais e preditas feitas por um classificador, como mostrado na Tabela \ref{tab:matriz_de_confusao}. Ela permite que a partir de seus elementos diversas medidas de desempenho sejam calculadas.

\begin{table}[ht]
  \caption{Matriz de confus�o para um classificador de duas classes.}
  \label{tab:matriz_de_confusao}
  \begin{center}
    \begin{tabular}{l|l|c|c|}

    \multicolumn{2}{c}{} & \multicolumn{2}{c}{Classe verdadeira} \\
    \cline{3-4}
    \multicolumn{2}{c|}{} & Positivo & Negativo \\
    \cline{2-4}
    \multirow{2}{*}{Classe predita} & Positivo & VP & FP \\
    \cline{2-4}
    & Negativo & FN & VN \\
    \cline{2-4}
    
    \end{tabular}
  \end{center}
\end{table}

Um verdadeiro positivo (VP) � um evento predito como pertencente � classe positiva, sendo este realmente pertencente � classe positiva. Um falso negativo (FN) acontece quando o evento predito como pertencente da classe negativa pertence � classe positiva. Analogamente, o verdadeiro negativo (VN) � um evento da classe negativa predito corretamente e o falso positivo (FP) um evento da classe positiva predito de forma incorreta.

No contexto desse trabalho, os elementos da matriz de confus�o podem ser descritos como:

\begin{itemize}
  \item Verdadeiro positivo (VP): mais um ve�culo foi contabilizado quando o mesmo adentrou a regi�o de contagem;
  \item Falso negativo (FN): mais um ve�culo n�o foi contabilizado quando o mesmo adentrou a regi�o de contagem;
  \item Verdadeiro negativo (VN): mais um ve�culo n�o foi contabilizado quando nenhum ve�culo adentrou a regi�o de contagem em uma janela de tempo;
  \item Falso positivo (FP): mais um ve�culo foi contabilizado quando nenhum ve�culo adentrou a regi�o de contagem.
\end{itemize}

Os eventos verdadeiro negativos (VN) foram contados utilizando uma janela de tempo, que corresponde ao tempo m�dio em que um ve�culo leva para percorrer a cena. Essa intervalo foi medido empiricamente para cada amostra de video.

Algumas medidas de desempenho podem ser calculadas a partir dos elementos da matriz de confus�o. S�o elas: Especificidade (Equa��o \ref{eq:especificidade}), Valor Preditivo Negativo (Equa��o \ref{eq:valor_preditivo_negativo}), Precis�o (Equa��o \ref{eq:precisao}), \textit{Recall} (Equa��o \ref{eq:recall}), Acur�cia (Equa��o \ref{eq:acuracia}) e \textit{F-Measure} (Equa��o \ref{eq:f_measure}) \citep{powers07evaluation}.

\begin{equation}
  \label{eq:especificidade}
  E=\dfrac{VN}{VN+FP}
\end{equation}

\begin{equation}
  \label{eq:valor_preditivo_negativo}
  VPN=\dfrac{VN}{VN+FN}
\end{equation}

\begin{equation}
  \label{eq:precisao}
  P=\dfrac{VP}{VP+FP}
\end{equation}

\begin{equation}
  \label{eq:recall}
  R=\dfrac{VP}{VP+FN}
\end{equation}

\begin{equation}
  \label{eq:acuracia}
  A=\dfrac{VP+VN}{VP+FP+FN+VN}
\end{equation}

\begin{equation}
  \label{eq:f_measure}
  FM=\dfrac{2*R*P}{R+P}
\end{equation}

% section matriz_de_confus_o (end)

\section{�ndice Kappa (K)} % (fold)
\label{sec:_ndice_kappa_}

O �ndice Kappa � utilizado como uma medida apropriada da exatid�o por representar inteiramente a matriz de confus�o. Ele toma todos os elementos da matriz ao inv�s de apenas aqueles que retratam a quantidade de classifica��es verdadeiras, o que ocorre quando se calcula a exatid�o global da classifica��o \citep{Jnl605963009}.

O �ndice Kappa pode ser encontrado com base na Equa��o \ref{eq:kappa}:

\begin{equation}
  \label{eq:kappa}
  K=\dfrac{\theta_{1}-\theta_{2}}{1-\theta_{2}}\text{,}
\end{equation}

\noindent onde

\begin{equation}
  \label{eq:theta_1}
  \theta_{1}=\dfrac{VP+VN}{VP+FP+FN+VN}\text{,}
\end{equation}

\noindent e

\begin{equation}
  \label{eq:theta_2}
  \theta_{2}=\dfrac{\alpha+\beta}{\gamma^{2}}\text{,}
\end{equation}

\noindent onde $\alpha=(VP+FN)*(VP+FP)$, $\beta=(VN+FN)*(VN+FP)$ e $\gamma=VP+VN+FP+FN$.

O n�vel de exatid�o obtido foi classificado conforme a Tabela \ref{tab:indice_kappa}, seguindo o estabelecido por \citeauthor{landis1977measurement} \citep{landis1977measurement}.

\begin{table}[ht]
  \caption{N�vel de exatid�o de uma contagem, conforme o valor do �ndice Kappa.}
  \label{tab:indice_kappa}
  \begin{center}
    \begin{tabular}{cc}
    \toprule
    \textbf{�ndice Kappa (K)} & \textbf{Qualidade} \\
    \midrule
      $K < 0.2$ & Ruim \\
      $0.2 \leq K < 0.4$ & Razo�vel \\
      $0.4 \leq K < 0.6$ & Bom \\
      $0.6 \leq K < 0.8$ & Muito bom \\
      $K \geq 0.8$ & Excelente \\
    \bottomrule
    \end{tabular}
  \end{center}
\end{table}

% section _ndice_kappa_ (end)

\section{Resultados finais} % (fold)
\label{sec:resultados_finais}

% Na Tabela \ref{tab:resultados_contagem} s�o apresentados os resultados obtidos pela contagem via m�todos manual e autom�tico, indicando o percentual de acerto para cada amostra. � importante destacar que o resultado da contagem autom�tica representa o total de ve�culos identificados na amostra, incluindo falsos positivos e descartando objetos detectados ou rastreados incorretamente. Portanto, os percentuais de acerto s�o medi��es estat�sticas, que representam apenas o desvio existente entre o valor real e o valor medido.

% \begin{table}[ht]
%   \caption{Resultados de contagem volum�trica.}
%   \label{tab:resultados_contagem}
%   \begin{center}
%     \begin{tabular}{l|cccc}
%     \toprule
%     \textbf{Nome do v�deo} & \textbf{Resolu��o} & \textbf{Manual} & \textbf{Autom.} & \textbf{\% acerto}\\
%     \midrule
%       carlos\_luz\_centro\_vga & $ 640\times 480 $ & 164 & 156 & 95,12 \\
%       carlos\_luz\_centro\_hd\_resized & $ 640\times 360 $ & 150 & 153 & 98,04 \\
%       carlos\_luz\_centro\_hd & $ 1280\times 720 $ & 150 & 145 & 96,67 \\
%       carlos\_luz\_pampulha\_vga\_1 & $ 640\times 480 $ & 208 & 173 & 83,17 \\
%       carlos\_luz\_pampulha\_vga\_2 & $ 640\times 480 $ & 169 & 138 & 81,66 \\
%       carlos\_luz\_pampulha\_hd\_2\_resized & $ 640\times 360 $ & 201 & 175 & 87,06 \\
%       carlos\_luz\_pampulha\_hd\_2 & $ 1280\times 720 $ & 201 & 170 & 84,58 \\
%     \bottomrule
%     \end{tabular}
%   \end{center}
% \end{table}

% Percebe-se que para uma mesma cena, a varia��o da resolu��o do v�deo n�o afeta de forma siginificativa o percentual de acerto, mostrando que v�deos de alta resolu��o n�o trazem vantagens nesse tipo de aplica��o. J� a mudan�a da cena alterou o resultado. Nas amostras do tr�fego no sentido centro o percentual de acerto foi superior a 95\%, enquanto que as imagens do outro sentido n�o ultrapassaram os 90\%. 

Analisando os v�deos � poss�vel identificar v�rios ve�culos de grande porte trafegando no sentido Pampulha, muitos saindo da F�brica da Coca-Cola ali localizada. Esses objetos com �rea muito grande causam problemas na etapa de subtra��o de \textit{background}, como mostrado na Figura \ref{fig:problema_veiculo_grande}, interferindo por alguns \textit{frames} no processo de detec��o de \textit{blobs} e gerando oclus�o em ve�culos menores. 

% Esse tipo de tr�fego na cena fez com que o resultado da contagem autom�tica ficasse sempre abaixo da refer�ncia em todas as amostras sentido Pampulha, justificando o percentual de acerto baixo.

\begin{figure}[ht]
  \begin{center}
    \begin{subfigure}[b]{.49\textwidth}
      \begin{center}
        \includegraphics[width=1\linewidth]{imgs/problema_veiculo_grande.png}
      \end{center}
      \caption{}
      \label{fig:problema_veiculo_grande}
    \end{subfigure}
    \begin{subfigure}[b]{.49\textwidth}
      \begin{center}
        \includegraphics[width=1\linewidth]{imgs/problema_veiculo_junto.png}
      \end{center}
      \caption{}
      \label{fig:problema_veiculo_junto}
    \end{subfigure}
  \end{center}
  \caption{Problemas encontrados no processamento de imagens. (a) A opera��o de subtra��o de \textit{background} fica comprometida quando ve�culos com �rea muito grande aparecem na cena. (b) Ve�culos pr�ximos s�o detectados como um �nico objeto, definindo apenas um \textit{keypoint}.}
  \label{fig:problemas}
\end{figure}

Outro tipo de problema acontece na etapa de detec��o de \textit{blobs}, ilustrado na Figura \ref{fig:problema_veiculo_junto}. Nessa situa��o, dois ve�culos pr�ximos foram identificados como um �nico objeto e apenas um \textit{keypoint} foi definido. Portanto, apenas um \textit{tracker} foi criado e um �nico ve�culo contado nesse caso.

Para avaliar o desempenho computacional do m�todo adotado, mediu-se o tempo de execu��o do algoritmo para cada uma das amostras, como mostra a Tabela \ref{tab:resultados_tempo}. Percebe-se que a resolu��o do v�deo est� diretamente relacionada ao tempo gasto no processamento. Os v�deos em HD foram processados com mais de tr�s vezes o tempo de processamento dos v�deos em VGA, inviabilizando a utiliza��o dessa configura��o em aplica��es de tempo real. Os v�deos redimensionados, com resolu��o $ 640\times 360 $, gastaram menos tempo de processamento do que sua pr�pria dura��o, indicando que a contagem em tempo real nesse caso seria vi�vel.

Com intuito de manter o algoritmo simples e o mais gen�rico poss�vel, os \textit{frames} s�o processados por completo, sem considerar regi�es de interesse. A defini��o de regi�es de interesse, tamb�m conhecidas como ROI's\footnote{\textit{Region of interest}}, � um recurso muito utilizado em vis�o computacional. Quando usado, apenas as por��es da imagem que contenham informa��es relevantes s�o consideradas no processamento, reduzindo o tempo e o custo computacional.

\begin{table}[ht]
  \caption{Tempo real gasto na execu��o do m�todo de contagem.}
  \label{tab:resultados_tempo}
  \begin{center}
    \begin{tabular}{lccc}
    \toprule
    \textbf{Nome do v�deo} & \textbf{Resolu��o} & \textbf{Dura��o} & \textbf{Tempo gasto} \\
    \midrule
      carlos\_luz\_centro\_vga & $ 640\times 480 $ & 00:05:23 & 00:05:47 \\
      carlos\_luz\_centro\_hd\_resized & $ 640\times 360 $ & 00:05:25 & 00:05:16 \\
      carlos\_luz\_centro\_hd & $ 1280\times 720 $ & 00:05:29 & 00:17:48 \\
      carlos\_luz\_pampulha\_vga\_1 & $ 640\times 480 $ & 00:05:43 & 00:06:04 \\
      carlos\_luz\_pampulha\_vga\_2 & $ 640\times 480 $ & 00:05:19 & 00:05:33 \\
      carlos\_luz\_pampulha\_hd\_2\_resized & $ 640\times 360 $ & 00:06:19 & 00:05:38 \\
      carlos\_luz\_pampulha\_hd\_2 & $ 1280\times 720 $ & 00:06:21 & 00:19:00 \\
    \bottomrule
    \end{tabular}
  \end{center}
\end{table}

% section resultados_finais (end)




























% chapter testes_e_resultados (end)

\chapter{Conclusão} % (fold)
\label{cha:conclus_o}
% !TEX root = pfc.tex

Baseado nos estudos das t�cnicas de processamento de imagens e trabalhos de vis�o computacional com objetivos semelhantes, foi criado um m�todo de simples implementa��o que realiza a segmenta��o e o acompanhamento de objetos, tornando assim poss�vel a contagem volum�trica dos mesmos. Em todas as amostras, mostradas na Tabela \ref{tab:resultados_contagem}, a contagem realizada se mostrou satisfat�ria.

O desempenho do algoritmo para as amostras em HD foi baixo, inviabilizando a execu��o em tempo real. No ambiente de testes utilizado, descrito na Se��o \ref{sec:testes}, apenas os v�deos com resolu��o $640\times 360$ poderiam ser utilizados para processamento em tempo real. Com um poder de processamento maior, provavelmente seria poss�vel realizar a contagem em tempo real nos v�deos com resolu��o $640\times 480$.

Embora j� existam diversos m�todos para a realiza��o de contagem volum�trica de ve�culos, como descrito no Cap�tulo \ref{cha:contagem_de_ve_culos_e_monitoramento_do_tr_fego}, geralmente eles s�o muito caros e causam transtornos na via durante sua instala��o e manuten��o. O trabalho em quest�o prop�e um m�todo n�o-invasivo, de baixo custo e simples implementa��o, capaz de obter medi��es de boa precis�o.

\section{Limita��es} % (fold)
\label{sec:pontos_negativos_do_m_todo}

V�rios aspectos de altera��o din�mica da cena interferiram no processamento digital. O algoritmo de subtra��o de \textit{background}, por exemplo, apresentou problemas quando ve�culos de grande porte aparecem nas imagens, como mostra a Figura \ref{fig:problema_veiculo_grande}. 

Como n�o foi desenvolvido um m�todo para identificar se o objeto em movimento � um veiculo ou n�o, alguns elementos indesej�veis foram considerados na contagem. Ao mesmo tempo, devido � proximidade, a detec��o de \textit{blobs} falhou e alguns ve�culos n�o foram contabilizados, como mostrado na Figura \ref{fig:problema_veiculo_junto}.

% section pontos_negativos_do_m_todo (end)

\section{Trabalhos futuros} % (fold)

Como poss�veis trabalhos futuros, podem ser apontados:

\begin{itemize}
  \item Defini��o de uma uma regi�o de interesse baseada na �rea com maior movimenta��o nas imagens;
  \item Desenvolvimento de um m�todo para contagem de ve�culos no per�odo noturno ou em locais com baixa ilumina��o;
  \item Cria��o de t�cnicas para classifica��o de ve�culos quanto ao tamanho;
  \item Contagem de ve�culos em cruzamentos, com deslocamentos em v�rias dire��es e sentidos;
  \item Estima��o do volume do tr�fego em vias congestionadas.
\end{itemize}

\label{sec:trabalhos_futuros}

% section trabalhos_futuros (end)

% chapter conclus_o (end)

% Aqui vem a parte da bibliografia: use o comando \ppgccbibliography indicando
% apenas o nome do arquivo .bib (sem a extensão).
\ppgccbibliography{bibfile}


% Este comando encapsula o conjunto de apêndices. A sua função é fazer com que
% a numeração dos apêndices seja feita com letras maiúsculas (A, B, C, etc.) e
% a palavra "Apêndice" anteceda as entradas no Sumário.
% \begin{appendices}

% Para cada apêndice, um \chapter
% \chapter{Um apêndice}
% \chapter{Outro apêndice}

% Fim dos apêndices (usar apenas depois do último apêndice)
% \end{appendices}


% Este comando encapsula o conjunto de anexos. A sua função é fazer com que a
% numeração dos anexos seja feita com letras maiúsculas (A, B, C, etc.) e a
% palavra "Anexo" anteceda as entradas no Sumário.
% \begin{attachments}

% Para cada anexo, um \chapter
% \chapter{Um anexo}
% \chapter{Outro anexo}

% Fim dos anexos (usar apenas depois do último anexo)
% \end{attachments}


\end{document}
