% !TEX root = pfc.tex

% Baseado nos estudos das técnicas de processamento de imagens e trabalhos de visão computacional com objetivos semelhantes, foi criado um método de simples implementação que realiza a segmentação e o acompanhamento de objetos, tornando assim possível a contagem volumétrica dos mesmos. Em todas as amostras, mostradas na Tabela \ref{tab:resultados_contagem}, a contagem realizada se mostrou satisfatória.

% O desempenho do algoritmo para as amostras em HD foi baixo, inviabilizando a execução em tempo real. No ambiente de testes utilizado, descrito na Seção \ref{sec:testes}, apenas os vídeos com resolução $640\times 360$ poderiam ser utilizados para processamento em tempo real. Com um poder de processamento maior, provavelmente seria possível realizar a contagem em tempo real nos vídeos com resolução $640\times 480$.

Ferramentas para análise do tráfego e planejamento viário ainda são objetos de estudo de diversos grupos de pesquisadores. O uso de tecnologias de simples implementação, instalação e baixo custo justificam e motivam a realização de trabalhos como esse.

O objetivo principal foi desenvolver um método de contagem volumétrica que auxiliasse na análise das condições do tráfego urbano. Isso foi feito de maneira simples, utilizando ferramentas de visão computacional já consolidadas. Ao final, através da análise dos resultados e índices de desempenho, determinou-se a qualidade do método.

O método proposto foi dividido em seis etapas: entrada de dados, pré-processamento, subtração de \textit{background}, binarização, detecção de \textit{blobs} e rastreamento e contagem. As cinco primeiras utilizam operações de visão computacional para extrair das imagens a posição e tamanho dos objetos em movimento. A última etapa utiliza esses pontos para implementar o rastreamento e contagem dos veículos em cena.

A metodologia proposta mostrou-se satisfatória diante das seguintes premissas: a cena possui boa iluminação, com pouca variação ao longo do tempo; não existem oclusões parciais ou totais entre veículos; a câmera não sofre grandes vibrações ou movimentações.

O objetivo principal do trabalho foi alcançado, já que foi feita a contagem volumétrica automática de veículos adotando uma metodologia simples. Além disso, o método obteve precisão média de $0.96$, acurácia média de $0.86$ e índice Kappa médio de $0.66$ (muito bom, segundo escala definida na Tabela \ref{tab:indice_kappa}). A contagem volumétrica da via no sentido Centro obteve Acurácia e índice Kappa superiores à contagem no sentido Pampulha, onde o fluxo de ônibus e caminhões é maior, contrariando as premissas previamente definidas. A acurácia alcançada é suficiente para o propósito deste trabalho.

% Embora já existam diversos métodos para a realização de contagem volumétrica de veículos, como descrito no Capítulo \ref{cha:contagem_de_ve_culos_e_monitoramento_do_tr_fego}, geralmente eles são muito caros e causam transtornos na via durante sua instalação e manutenção. O trabalho em questão propõe um método não-invasivo, de baixo custo e simples implementação, capaz de obter medições de boa precisão.

\section{Limitações} % (fold)
\label{sec:pontos_negativos_do_m_todo}

Aspectos de alteração dinâmica da cena interferiram no processamento digital. O algoritmo de subtração de \textit{background} apresentou problemas quando veículos de grande porte aparecem nas imagens, como mostra a Figura \ref{fig:problema_veiculo_grande}. Nesses casos, o modelo do \textit{background} muda subitamente na presença de um veículo de tamanho desproporcional à cena, gerando imperfeições na segmentação do \textit{foreground}. Além disso, movimentações na câmera causadas por trepidações da passarela também interferiram na subtração de \textit{background}, comprometendo as etapas subsequentes.

% Ainda com relação às movimentações da câmera, alguns elementos indesejáveis foram considerados na contagem. Ao mesmo tempo, devido à proximidade, a detecção de \textit{blobs} falhou e alguns veículos não foram contabilizados, como mostrado na Figura \ref{fig:problema_veiculo_junto}.

% section pontos_negativos_do_m_todo (end)

\section{Trabalhos futuros} % (fold)

Como possíveis trabalhos futuros, podem ser apontados:

\begin{itemize}
  \item Desenvolvimento de um método para contagem de veículos no período noturno, em locais com baixa iluminação ou alta variação de luminosidade;
  % \item Definição de uma região de interesse baseada na área com maior movimentação nas imagens;
  \item Criação de técnicas para classificação de veículos quanto ao tamanho;
  \item Contagem de veículos em cruzamentos, com deslocamentos em várias direções e sentidos;
  \item Estimação do volume do tráfego em vias congestionadas.
\end{itemize}

\label{sec:trabalhos_futuros}

% section trabalhos_futuros (end)