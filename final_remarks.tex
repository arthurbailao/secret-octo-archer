% !TEX root = pfc.tex

% Baseado nos estudos das técnicas de processamento de imagens e trabalhos de visão computacional com objetivos semelhantes, foi criado um método de simples implementação que realiza a segmentação e o acompanhamento de objetos, tornando assim possível a contagem volumétrica dos mesmos. Em todas as amostras, mostradas na Tabela \ref{tab:resultados_contagem}, a contagem realizada se mostrou satisfatória.

O desempenho do algoritmo para as amostras em HD foi baixo, inviabilizando a execução em tempo real. No ambiente de testes utilizado, descrito na Seção \ref{sec:testes}, apenas os vídeos com resolução $640\times 360$ poderiam ser utilizados para processamento em tempo real. Com um poder de processamento maior, provavelmente seria possível realizar a contagem em tempo real nos vídeos com resolução $640\times 480$.

Embora já existam diversos métodos para a realização de contagem volumétrica de veículos, como descrito no Capítulo \ref{cha:contagem_de_ve_culos_e_monitoramento_do_tr_fego}, geralmente eles são muito caros e causam transtornos na via durante sua instalação e manutenção. O trabalho em questão propõe um método não-invasivo, de baixo custo e simples implementação, capaz de obter medições de boa precisão.

\section{Limitações} % (fold)
\label{sec:pontos_negativos_do_m_todo}

Vários aspectos de alteração dinâmica da cena interferiram no processamento digital. O algoritmo de subtração de \textit{background}, por exemplo, apresentou problemas quando veículos de grande porte aparecem nas imagens, como mostra a Figura \ref{fig:problema_veiculo_grande}. Além disso, movimentações na câmeras causadas por trepidações da passarela quando veículos de grande porte transitavam na via, também geraram ruídos na subtração de \textit{background}, comprometendo as etapas subsequentes.

Como não foi desenvolvido um método para identificar se o objeto em movimento é um veiculo ou não, alguns elementos indesejáveis foram considerados na contagem. Ao mesmo tempo, devido à proximidade, a detecção de \textit{blobs} falhou e alguns veículos não foram contabilizados, como mostrado na Figura \ref{fig:problema_veiculo_junto}.

% section pontos_negativos_do_m_todo (end)

\section{Trabalhos futuros} % (fold)

Como possíveis trabalhos futuros, podem ser apontados:

\begin{itemize}
  \item Minimizar o prejuízo causado pela movimentação da câmera pelo uso de pontos fiduciais;
  \item Definição de uma uma região de interesse baseada na área com maior movimentação nas imagens;
  \item Desenvolvimento de um método para contagem de veículos no período noturno ou em locais com baixa iluminação;
  \item Criação de técnicas para classificação de veículos quanto ao tamanho;
  \item Contagem de veículos em cruzamentos, com deslocamentos em várias direções e sentidos;
  \item Estimação do volume do tráfego em vias congestionadas.
\end{itemize}

\label{sec:trabalhos_futuros}

% section trabalhos_futuros (end)